% -*- coding: utf-8 -*-
%%
%%  本模板可以使用以下两种方式编译:
%%     1. PDFLaTeX
%%     2. XeLaTeX [推荐]
%%  注意:
%%    1. 在改变编译方式前应先删除 *.toc 和 *.aux 文件,
%%       因为不同编译方式产生的辅助文件格式可能并不相同。

\documentclass{cumcmart}
% \documentclass[nocover]{cumcmart}%%%切换到无封面的版本,有些区域不允许前面的承诺页用pdf格式,可以用此去掉。



\begin{document}

\xuanti{A}
%\school命令用于在承诺书上显示学校名称。按要求,此处应填写全称
\school{长江大学}
%以下命令分别显示队员及指导教师姓名
\numbers{2017000}%参赛报名号
\authorone{成员一}
\authortwo{成员二}
\authorthree{成员三}
\advisor{数模指导组}

%\theyear{2017}
\theday{07}%填写当月的具体日期

\title{Title}

\maketitle

\begin{cnabstract}%此处没有采用sbstract命名,是为了将来如果要加入英文摘要时扩展的方便


\cnkeywords{符号秩检验,方差检验因子分析,聚类分析,}
\end{cnabstract}

\newpage
%\tableofcontents\newpage%增加目录,要不要都可以。不想要的话,就在本行前加“%”(英文的百分号)


\section{问题重述}
确定葡萄酒质量时一般是通过聘请一批有资质的评酒员进行品评。每个评酒员在对葡萄酒进行品尝后对其分类指标打分,然后求和得到其总分,从而确定葡萄酒的质量。酿酒葡萄的好坏与所酿葡萄酒的质量有直接的关系,葡萄酒和酿酒葡萄检测的理化指标会在一定程度上反映葡萄酒和葡萄的质量。

首先需要分析两组评酒员的评价结果有无显著性差异,如果有则需要确定哪一组评酒员的结果更可信;然后根据酿酒葡萄的理化指标和葡萄酒的质量,对这些酿酒葡萄进行分级;根据酿酒葡萄与葡萄酒的理化指标,寻找其中的联系;分析酿酒葡萄和葡萄酒的理化指标对葡萄酒质量的影响,并论证能否用葡萄和葡萄酒的理化指标来评价葡萄酒的质量。


\section{建模分析}

\subsection{模型假设}
对于葡萄和葡萄酒的理化指标与葡萄和葡萄酒的质量所建立的模型模型,我们提出了如下的合理假设:
\begin{enumerate}
\item 未考虑到的酿酒葡萄理化指标不会影响葡萄酒的质量;
\item 被检验的葡萄能都够代表该葡萄种类的特性;
\item 附件中给出的葡萄和葡萄酒理化指标都准确可靠;
\item 
\end{enumerate}

\subsection{记号说明}
\begin{table}[!htbp]
    \centering
    \begin{tabular}{cl}
    \toprule
    \multicolumn{2}{c}{\large 模型记号说明}\\
    \midrule
    ${c_{ij}}$ &  i产品在j设备上加工的件数 \\
    ${e_{ij}}$ &  i产品在j设备上加工所需要的工时 \\
    ${a_i}$    &  i产品每件获得的利润 \\
    ${b_j}$    &  j设备最大负荷工时 \\
    ${d_j}$    &  j设备满负荷时需要的费用 \\
    \bottomrule
    \end{tabular}
    \caption{模型记号说明}
\end{table}

\subsection{建立模型}

    
    \[ %\begin{equation}
    s.t.
    \left\{  
    \begin{array}{ll}  
    \sum\limits_{i=1}^{2}{c_{ij}} = \sum\limits_{i=3}^{5}{c_{ij}} & j  = 1,2,3  \\ 
     \sum\limits_{i=1}^{3}{c_{ij}e_{ij}} \leqslant b_{i} & i = 1,2 \cdots 5 \\
    \end{array}  
    \right.
    \] %\end{equation} 

\subsection{模型求解和分析}
 

\subsection{模型评价}
\subsubsection{模型优点}
1)	

2)	

3)	

\subsubsection{模型缺点}
1)	

2)	


%   \begin{figure}
%   \centering
%   \includegraphics[width=.6\textwidth]{fig1}
%   \caption{发生事故时车流饱和状态图示}
%   \end{figure}

% \begin{thebibliography}{10}
% \bibitem{1} \url{http://bbs.chinatex.org}
% \bibitem{2} \url{http://www.chinatex.org}
% \bibitem{3} Alpha Huang, \textbf{latex-notes-zh-cn}, 2014.
% \bibitem{lf}M.R.C. van Dongen,\textbf{\LaTeX-and-Friends}, 2013.
% \bibitem{figure}Keith Reckdahl,\textbf{Using Import graphics in \LaTeXe}, 1997.
% \bibitem{HM}Addison Wesley,\textbf{Higher Mathematics}, 下载地址如下\\ \url{http://media.cism.it/attachments/ch8.pdf}
% \end{thebibliography}


\newpage
\appendix
\section*{附 \quad 录}


\end{document}
